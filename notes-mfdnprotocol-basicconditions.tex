% ================================================================================
\documentclass[12pt]{../document-templates/papers/one-column-mydashie/mydashie}
% ================================================================================
\usepackage[left=1in,right=1in, top=1.2in,bottom=1.2in]{geometry}
\usepackage{../document-templates/papers/one-column-mydashie/mydashie}
\usepackage{times}
\usepackage{amssymb,amsthm,latexsym,amsmath,epsfig,pgf}

\usepackage{graphicx}
\usepackage{comment}
\usepackage{url}
\usepackage{hyperref}
% ================================================================================
\firstpage{1}
% ================================================================================
\runauth{
Notes: Basic conditions for a MfDN-Protocol
\hspace{2ex} $\arrowvert$\hspace{2ex}
Carolin Z\"obelein}
% ================================================================================
\newtheorem{theorem}{Theorem}[section]
\newtheorem*{theorem A}{Theorem A}
\newtheorem*{theorem B}{N\"olker's Theorem}
\newtheorem{lemma}{Lemma}[section]
\newtheorem{proposition}{Proposition}[section]
\newtheorem{corollary}{Corollary}[section]
\newtheorem{problem}{Problem}
\newtheorem*{question}{Question}
\newtheorem {conjecture}{Conjecture}
\theoremstyle{remark}
\newtheorem{remark}{Remark}[section]
\theoremstyle{remark}
\newtheorem{remarks}{Remarks}
% ================================================================================
\begin{document}
% ================================================================================
\begin{frontmatter}
% ================================================================================
\title{Notes: Basic conditions for a mathematical forced decentralized network protocol\\
(MfDN-Protocol)}
% ================================================================================
\author[label1]{Carolin Z\"obelein\footnote{\url{https://research.carolin-zoebelein.de}}\footnote{The author believes in the importance of the independence of research and is funded by the public community. If you also believe in this values, you can find ways for supporting the author's work here: \url{https://research.carolin-zoebelein.de/funding.html}}, id: \textrm{notes\_0006}}

%\address[label1]{\small Institute}
\address[label1]{\small Independent mathematical scientist,\\
	Josephsplatz 8, 90403 N\"urnberg,\\
	Germany

\vspace*{2.5ex} 
 {\normalfont contact@carolin-zoebelein.de\\
	PGP: D4A7 35E8 D47F 801F 2CF6 2BA7 927A FD3C DE47 E13B}
 }
% ================================================================================
\begin{abstract}
A mathematical forced decentralized network protocol needs a set of basic conditions which has to be considered. To get more clear about it, we discuss the differences of this concept regarding to classical decentralized systems and give a first starting point for further design steps.
\end{abstract}
% ================================================================================
\begin{keyword}
% Separate keyword by \sep
Decentralization \sep Networks \sep Protocol \sep Mathematical \sep Privacy

% Write the classification number
ACM Computing Classification: Networks - Network protocol design.
\end{keyword}
% ================================================================================
\end{frontmatter}
% ================================================================================
% --------------------------------------------------------------------------------
\section{Introduction}
\label{s:introduction}
% --------------------------------------------------------------------------------
In this notes, we will discuss the main basic conditions which are necessary for a mathematical forced decentralized network protocol, short \textit{MfDN-Protocol}.
% ================================================================================
% --------------------------------------------------------------------------------
\section{What does MfDN-Protocol mean?}
\label{s:whatdoesmfdnprotocolmean}
% --------------------------------------------------------------------------------
Decentralization is given by distributing computation power, or by 'sharing splited' data or information with each other on a technical, computer science level. All of this could theoretically be happen also on only one device. Also the common shared data, can be centralized without any problem on only a few or even one machine.

Hence, \textit{Mathematical forced dececentralization} means a protocol which follows a mathematical structure which needs mandatory decentralization to be able to work. To understand this concept better, we want to introduce it by having a look at the differences between a centralized service, a classical decentralized service and a mathematical forced decentralized service, at first.
% --------------------------------------------------------------------------------
\subsection{Centralized service}
\label{ss:centralizedservice}
% --------------------------------------------------------------------------------
\textit{Situation.} The service is maintained by a single party on a centralized computer network.

\vspace{0.3cm}
\textit{Main design points.}
\begin{enumerate}
	\item One single party maintains the service
	\item The service runs on a centralized computer network
	\item Every user of the service have to connect to this centralized network
	\item Data is saved centralized
	\item Traffic is centralized to the service computer network
\end{enumerate}

\vspace{0.3cm}
\textit{Disadvantages.} The service itself, the collected data and the generated traffic, the user traffic as well as the service traffic, is centralized and hence in the power of the single party who runs the service. This makes acquisition of data and censorship of information very easy. Additionally, centralized networks have limited scalability.
% --------------------------------------------------------------------------------
\subsection{Classical decentralized service}
\label{ss:classicaldecentralizedservice}
% --------------------------------------------------------------------------------
\textit{Situation.} The service is maintained by a bunch of parties on a decentralized computer network.

\vspace{0.3cm}
\textit{Main design points.}
\begin{enumerate}
	\item Several parties maintain the service
	\item The service runs in a decentralized computer network
	\item Every user of the service connects to the decentralized network
	\item Data can be saved decentralized
	\item Traffic is centralized to the members of the whole decentralized computer network
\end{enumerate}

\vspace{0.3cm}
\textit{Disadvantages.} Even if we have a decentralized service, a centralization on a deeper level is still possible. Services itself, can be designed as centralized services itself, only running as several instances on several computers, data can, but not have, to be saved decentralized and even if users connect to a decentralized service, their individuell traffic can still be distributed within the network in the same way like within a centralized network, without mandatory anonymization and identifiable information about the origin. Their individual traffic can be still centralized, too.
% --------------------------------------------------------------------------------
\subsection{Mathematical forced decentralized service}
\label{ss:mathematicalforceddecentralizedservice}
% --------------------------------------------------------------------------------
\textit{Situation.} A mathematical forced decentralized network protocol is maintained by a bunch of parties on a decentralized computer network. The service is run on the top of this protocol.

\vspace{0.3cm}
\textit{Main design points.}
\begin{enumerate}
	\item Several parties maintain the protocol network for the service
	\item The service runs on the top of the mathematical forced decentralized network protocol in a decentralized computer network
	\item Every user of the service connect to this protocol which coordinates the decentralization itself
	\item All data is decentralized saved, organized by the protocol
	\item The protocol manages the partition of user and server traffic within the network
	\item The protocol is designed in such a way, that each service which runs on the top of it, can't be centralized, without loosing the ability to work. This forced decentralization is given by mathematical design aspects of the protocol, not by the technical, physical decentralization on computer level, alone.
\end{enumerate}

\vspace{0.3cm}
\textit{Disadvantages.} A given service has to be designed in such a way, that it is able to work on top of our given protocol (the protocol should support as much service designs as possible but it will be necessary that a service fulfills at least a minimum amount of design constraints.
% ================================================================================
% --------------------------------------------------------------------------------
\section{Critical design questions}
\label{s:criticaldesignquestions}
% --------------------------------------------------------------------------------
For our protocol, we have to take care of the following critical design questions:
\begin{enumerate}
	\item Defining the minimum amount of design contraints which a service has to fulfill
	\item How to take care of avoiding a 'fake' decentralization by e.g. running several virtual machines on the same centralized server?
	\item Best way of handling and coordinating network members, synchronization, concurrency control, ....	
	\item Best way of handling recovery problems because of user errors, application errors or partial system failures
	\item The given hardware and infrastrucutre have to be able to handle the amount of traffic, synchronization, ....
\end{enumerate}
% ================================================================================
% --------------------------------------------------------------------------------
\section{Main decentralization points}
\label{s:maindecentralizationpoints}
% --------------------------------------------------------------------------------
If we look at the mentioned concepts of networks, it crystallizes the following main decentralization points for our MfDN-Protocol:
\begin{enumerate}
	\item A mathematical forced decentralization of the running service
	\item A mathematical forced decentralization of data
	\item A mathematical forced decentralization of user and service traffic
\end{enumerate} 

We will more clarify this in the further research discussion notes. For the moment, we will simple keep this in mind, at first.
% ================================================================================
\begin{comment}
%% Notes

Critical design points/questions:
How to be sure that decentralization is not only 'simulated'?

The service can still be centralized on a single member of the network (every member of the network runs the same service)
Parts of each user access have to be separated, parted, decentralized
% --------------------------------------------------------------------------------
Decentralization is given by distributing computation power, or by 'sharing splited' data/information to each other
on a technical/computer science level
All of this could theoretically be happen also on one device
Also the common shared data, can be centralized without problem on a few or one machine

'Mathematical forced dececentralization' means a protocol which follows a mathematical structure
which needs nececcary decentralization to be able to work
\end{comment}
% ================================================================================
% --------------------------------------------------------------------------------
\section{Conclusion}
\label{s:conclusion}
% --------------------------------------------------------------------------------
After having a look at the basic conditions for our mathematicial forced decentralized protocol, we want to put this in a more clear design in the next notes. There we will discuss the basic protocol design and the deeper parts with the mathematical aspects of our work.
% ================================================================================
% --------------------------------------------------------------------------------
\section{Further Information}
\label{s:furtherinformation}
% --------------------------------------------------------------------------------
In the following, you can find some additional information related to this paper.
% --------------------------------------------------------------------------------
\subsection{Project}
\label{ss:project}
% --------------------------------------------------------------------------------
This paper belongs to the project \textit{Combsee} (former \textit{Decentralized privacy preserving search by mathematical design (DePPSMaD)}). It's belonging id is \texttt{project\_0055} and its project page is \href{https://research.carolin-zoebelein.de/Projects/2019/Combsee.html}
	{\nolinkurl{https://research.carolin-zoebelein.de/Projects/2019/Combsee.html}}.
% --------------------------------------------------------------------------------
\subsection{Funding}
\label{ss:funding}
% --------------------------------------------------------------------------------
This paper belongs to the project 0055 which is supported by a funding of \textit{NL NGI Zero Discovery} \url{https://nlnet.nl/discovery/}.
% ================================================================================
\section*{Acknowledgement} 
Special thanks to the individual donators for supporting this work.
% ================================================================================
%\section*{References}
% --------------------------------------------------------------------------------
%\newpage
%\clearpage
%\markboth{Bibliography}{Bibliography}
%\section*{Bibliography}
%\label{s:bibliography}
% --------------------------------------------------------------------------------
%\bibliographystyle{amsplain}
%\bibliographystyle{unsrtdin}
%\bibliographystyle{plain}

% TOOD
%\nocite{*}
%\bibliographystyle{unsrtdin}
%\bibliography{notes-mfdnprotocol-basicconditions}
% ================================================================================
\section*{License}
\label{s:license}
% --------------------------------------------------------------------------------
\begin{center}
	\includegraphics{by-nc-nd.png} \\
	\url{https://creativecommons.org/licenses/by-nc-nd/4.0/}
\end{center}
% ================================================================================
\end{document}
% ================================================================================
